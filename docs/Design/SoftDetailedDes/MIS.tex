\documentclass[12pt, titlepage]{article}

\usepackage{amsmath, mathtools}

\usepackage[round]{natbib}
\usepackage{amsfonts}
\usepackage{amssymb}
\usepackage{graphicx}
\usepackage{colortbl}
\usepackage{xr}
\usepackage{hyperref}
\usepackage{longtable}
\usepackage{xfrac}
\usepackage{tabularx}
\usepackage{float}
\usepackage{siunitx}
\usepackage{booktabs}
\usepackage{multirow}
\usepackage[section]{placeins}
\usepackage{caption}
\usepackage{fullpage}

\hypersetup{
bookmarks=true,     % show bookmarks bar?
colorlinks=true,       % false: boxed links; true: colored links
linkcolor=red,          % color of internal links (change box color with linkbordercolor)
citecolor=blue,      % color of links to bibliography
filecolor=magenta,  % color of file links
urlcolor=cyan          % color of external links
}

\usepackage{array}

\externaldocument{../../SRS/SRS}

%% Comments

\usepackage{color}

\newif\ifcomments\commentstrue %displays comments
%\newif\ifcomments\commentsfalse %so that comments do not display

\ifcomments
\newcommand{\authornote}[3]{\textcolor{#1}{[#3 ---#2]}}
\newcommand{\todo}[1]{\textcolor{red}{[TODO: #1]}}
\else
\newcommand{\authornote}[3]{}
\newcommand{\todo}[1]{}
\fi

\newcommand{\wss}[1]{\authornote{blue}{SS}{#1}} 
\newcommand{\plt}[1]{\authornote{magenta}{TPLT}{#1}} %For explanation of the template
\newcommand{\an}[1]{\authornote{cyan}{Author}{#1}}

%% Common Parts

\newcommand{\progname}{OAR} % PUT YOUR PROGRAM NAME HERE
\newcommand{\authname}{Hunter Ceranic} % AUTHOR NAMES                  

\usepackage{hyperref}
    \hypersetup{colorlinks=true, linkcolor=blue, citecolor=blue, filecolor=blue,
                urlcolor=blue, unicode=false}
    \urlstyle{same}
                                


\def\code#1{\texttt{#1}}

\begin{document}

\title{Module Interface Specification for OAR}

\author{Hunter Ceranic}

\date{March 15, 2024}

\maketitle

\pagenumbering{roman}

\section{Revision History}

\begin{tabularx}{\textwidth}{p{3cm}p{2cm}X}
\toprule {\bf Date} & {\bf Version} & {\bf Notes}\\
\midrule
March 8, 2024 & 1.0 & Initial Revision\\
March 15, 2024 & 1.1 & Changes made according to Dr. Smith's Initial Comments\\
April 10, 2024 & 1.2 & Changes made according to Comments from Primary and Secondary Reviewers\\
April 10, 2024 & 1.3 & Changes made according to Dr. Smith's Detailed Comments\\
\bottomrule
\end{tabularx}

~\newpage

\section{Symbols, Abbreviations and Acronyms}

See SRS Documentation \citep{SRS} at \url{https://github.com/cer-hunter/OAR-CAS741/blob/main/docs/SRS/SRS.pdf}

\newpage

\tableofcontents

\newpage

\pagenumbering{arabic}

\section{Introduction}

The following document details the Module Interface Specifications for
the OAR (Optical Alphabet Recognition) program. This document specifies how each module
interfaces with other parts of the program.

Complementary documents include the System Requirement Specifications
and Module Guide.  The full documentation and implementation can be
found at \url{https://github.com/cer-hunter/OAR-CAS741}. 

\section{Notation}

The structure of the MIS for modules comes from \citet{HoffmanAndStrooper1995},
with the addition that template modules have been adapted from
\cite{GhezziEtAl2003}.  The mathematical notation comes from Chapter 3 of
\citet{HoffmanAndStrooper1995}.  For instance, the symbol := is used for a
multiple assignment statement and conditional rules follow the form $(c_1
\Rightarrow r_1 | c_2 \Rightarrow r_2 | ... | c_n \Rightarrow r_n )$.

The following table summarizes the primitive data types used by OAR. 

\begin{center}
\renewcommand{\arraystretch}{1.2}
\noindent 
\begin{tabular}{l l p{7.5cm}} 
\toprule 
\textbf{Data Type} & \textbf{Notation} & \textbf{Description}\\ 
\midrule
character & char & a single symbol or digit\\
integer & $\mathbf{Z}$ & a number without a fractional component in (-$\infty$, $\infty$) \\
positive integer & $\mathbf{Z}_{+}$ & a positive integer ($\mathbf{Z}$) in ($0$, $\infty$) \\
unsigned 8-bit integer & $\mathbf{U}$ & a number without a fractional component in ($0$, $255$) \\
natural number & $\mathbf{N}$ & a number without a fractional component in [1, $\infty$) \\
real & $\mathbf{R}$ & any number in (-$\infty$, $\infty$)\\
positive real & $\mathbf{R}_{+}$ & any real number ($\mathbf{R}$) in ($0$, $\infty$) \\
\bottomrule
\end{tabular} 
\end{center}

\noindent
The specification of OAR uses some derived data types: sequences, strings,
tuples, and booleans. Sequences are lists filled with elements of the same data type. Strings
are sequences of characters. Tuples contain a list of values, potentially of
different types. Booleans can be represented in different ways but only have two possible values: true or false. In addition, 
OAR uses functions, which
are defined by the data types of their inputs and outputs. Local functions are
described by giving their type signature followed by their specification.

\section{Module Decomposition}

The following table is taken directly from the Module Guide document for this project.

\begin{table}[h!]
  \centering
  \begin{tabular}{p{0.3\textwidth} p{0.6\textwidth}}
  \toprule
  \textbf{Level 1} & \textbf{Level 2}\\
  \midrule
    
  {Hardware-Hiding Module} & ~\\
  \midrule
    
  \multirow{8}{0.3\textwidth}{Behaviour-Hiding Module}
    & Application Control \\
    & Graphics Display \\
    & Output Calculator \\
    & Input Data Read \\
    & Input Classifier \\
    & OAR Model Data \\
    & OAR Model Equations \\
    & OAR Model Training\\
    & OAR Model Testing\\
    \midrule
    
    \multirow{3}{0.3\textwidth}{Software Decision Module}
      & Confusion Matrix\\
      & Input Processing\\
      & Graphical User Interface \\
    \bottomrule
    
    \end{tabular}
  \caption{Module Hierarchy}
  \label{TblMH}
  \end{table}
  
\newpage

\section{MIS of Application Control Module} \label{ModuleAC} 

\subsection{Module}

\code{main} 

\subsection{Uses}

\begin{itemize}
  \item Graphics Display Module Specification (\ref{ModuleGD})
  \item Output Module Specification
\end{itemize}

\subsection{Syntax}

\subsubsection{Exported Constants}

None.

\subsubsection{Exported Access Programs}

\begin{center}
\begin{tabular}{p{2cm} p{4cm} p{4cm} p{2cm}}
\hline
\textbf{Name} & \textbf{In} & \textbf{Out} & \textbf{Exceptions} \\
\hline
\code{main} & - & - & - \\
\hline
\end{tabular}
\end{center}

\subsection{Semantics}

\subsubsection{State Variables}

None.

\subsubsection{Environment Variables}

\begin{itemize}
  \item Screen ($\mathbf{Z}_{+}$ for width and height in pixels)
\end{itemize}

\subsubsection{Assumptions}

The GUI Display is running and displayed without issue.

\subsubsection{Access Routine Semantics}

\noindent \code{main}():
\begin{itemize}
  \item transition: Connects the Output Calculator Module \ref{ModuleO} to the Graphics Display module \ref{ModuleGD}
\end{itemize}

\subsubsection{Local Functions}

None.

\section{MIS of Graphics Display} \label{ModuleGD} 

\subsection{Module}

\code{display}

\subsection{Uses}

\begin{itemize}
  \item Hardware-Hiding Module  
  \item Input Data Read Module (\ref{ModuleIDR})
  \item Output Calculator Module (\ref{ModuleO})
  \item Graphical User interface (GUI) Module (\ref{ModuleGUI})
\end{itemize}

\subsection{Syntax}

\subsubsection{Exported Constants}

\begin{itemize}
  \item \code{GUI\_BOXSIZE}: A value ($\mathbf{Z}_{+}$) describing both width and height (in pixels) used for the image
  display "box" (currently always a square)
\end{itemize}

\subsubsection{Exported Access Programs}

\begin{center}
\begin{tabular}{p{2cm} p{4cm} p{4cm} p{2cm}}
\hline
\textbf{Name} & \textbf{In} & \textbf{Out} & \textbf{Exceptions} \\
\hline
\code{display} & \code{inputImage} ($\mathbf{U}^{m \times n}$), \code{resultLabel} (String), \code{resultConf} (String) & \code{displayWindow} ($\mathbf{Z}_{+}^{m \times n}$), event handlers & \code{guiException} \\
\hline
\end{tabular}
\end{center}

\subsection{Semantics}

\subsubsection{State Variables}

\begin{itemize}
  \item inputImage ($\mathbf{U}^{m \times n}$): The processed input image and given by the Output Calculator Module \ref{ModuleO}
  \item resultLabel (String): The label output as given by the Output Calculator Module \ref{ModuleO} as a string.
  \item resultConf ($\mathbf{R}$): The confidence probability output as given by the Output Calculator Module \ref{ModuleO} as a string.
\end{itemize}

\subsubsection{Environment Variables}

\begin{itemize}
  \item Keyboard ($\mathbf{Z}_{+}$ for keycodes describing the key pressed)
  \item Mouse (Boolean for click state and  $\mathbf{Z}_{+}$ for cursor position)
  \item Screen ($\mathbf{Z}_{+}$ for width and height in pixels)
  \item \code{displayWindow} ($\mathbf{Z}_{+}$ for width and height in pixels) for the application interface
  \item \code{inputButton} (String for a file location) to provide an input image from the file system
\end{itemize}

\subsubsection{Assumptions}

\begin{itemize}
  \item The file system is able to read and provide the image file as specified by the user through an OS file-open dialog.
  Otherwise if the file is not found, denied access or cancelled, no changes should occur.
  \item The OS is able to provide basic text or number input user controls with some basic built-in validation, and
  is able to handle events from Human Interface Devices (HIDs such as mouse, keyboard or touchscreen).
\end{itemize}


\subsubsection{Access Routine Semantics}

\noindent \code{display}():
\begin{itemize}
\item transition: Sets up user control event handlers (i.e., mouse clicks or drag, button presses, text input change, ...) 
as needed for the user input. Calls the Input Data Read Module \ref{ModuleIDR} to accept a base image and Output Calculator Module \ref{ModuleO} to classify the input
image. The input image and output results are then pushed to the \code{displayWindow}.
\item exception: \code{guiException} when \code{ValueError} is raised by the program, or incorrect user input.
\end{itemize}

\subsubsection{Local Functions}

None.

\section{MIS of Output Calculator Module} \label{ModuleO} 


\subsection{Module}

\code{output}

\subsection{Uses}

\begin{itemize}
  \item Input Classifier Module Specification (\ref{ModuleIC})
\end{itemize}

\subsection{Syntax}

\subsubsection{Exported Constants}

\begin{itemize}
  \item \code{DEC\_FIXED}: Used for fixed decimal number length rounding (ex. "5.8923" at fixed length "2" results in "5.89")
\end{itemize}

\subsubsection{Exported Access Programs}

\begin{center}
\begin{tabular}{p{2cm} p{2cm} p{6cm} p{2cm}}
\hline
\textbf{Name} & \textbf{In} & \textbf{Out} & \textbf{Exceptions} \\
\hline
\code{output} & \code{baseImage} ($\mathbf{Z}^{m,n}$) & \code{displayImage} ($\mathbf{U}^{m \times n}$), \code{resultLabel} (String), \code{resultConf} (String) & - \\
\hline
\end{tabular}
\end{center}

\subsection{Semantics}

\subsubsection{State Variables}

\begin{itemize}
\item \code{labelData} (tuple):= outputs from \code{classify}(displayImage) containing the label (string), and confidence ()$\mathbf{R}_{+}$) in the prediction
\end{itemize}

\subsubsection{Environment Variables}

None.

\subsubsection{Assumptions}

The input image is valid.

\subsubsection{Access Routine Semantics}

\noindent \code{output}():
\begin{itemize}
\item output: \code{resultLabel}, \code{resultConf} := \code{labelData} (String)\\
\code{displayImage} := \code{input}(baseImage)
\end{itemize}

\subsubsection{Local Functions}

None.

\section{MIS of Input Data Read Module} \label{ModuleIDR} 

\subsection{Module}

\code{input}

\subsection{Uses}

\begin{itemize}
  \item Input Processing Module Specification (\ref{ModuleIP})
\end{itemize}

\subsection{Syntax}

\subsubsection{Exported Constants}

None.

\subsubsection{Exported Access Programs}

\begin{center}
\begin{tabular}{p{2cm} p{4cm} p{4cm} p{2cm}}
\hline
\textbf{Name} & \textbf{In} & \textbf{Out} & \textbf{Exceptions} \\
\hline
\code{input} & \code{inputPath} (String) & \code{inputImage} ($\mathbf{I}_{x,y}$) & \code{ValueError} \\
\hline
\end{tabular}
\end{center}

\subsection{Semantics}

\subsubsection{State Variables}
\begin{itemize}
\item \code{maxSize}: A value ($\mathbf{Z}_{+}$) describing both width and height (in pixels) for maximum acceptable 
size of the input image (currently a square).
\item \code{minSize}: A value ($\mathbf{Z}_{+}$) describing both width and height (in pixels) for minimum acceptable 
size of the input image (currently a square).
\item \code{modelSize}: The required size of the input image matrix to be used by the classification model $\mathbf{I}_{x,y}$.
\end{itemize}

\subsubsection{Environment Variables}

\begin{itemize}
  \item \code{baseImage}: The base input image in the form of a .BMP .JPG or .PNG file.
\end{itemize}

\subsubsection{Assumptions}

The \code{inputPath} location for the \code{baseImage} is valid, readable and accessible.

\subsubsection{Access Routine Semantics}

\noindent \code{input}(inputPath):
\begin{itemize}
\item output: \code{inputImage} := \code{preprocess}(baseImage) 
\item exception: \code{ValueError} if the size of the base image is outside of the range of 
\code{minSize} to \code{maxSize}, \code{ValueError} if the file type of the \code{baseImage} is not 
supported by the OAR Program (according to R\ref{R1})
\end{itemize}

\subsubsection{Local Functions}

None.


\section{MIS of Input Classifier Module} \label{ModuleIC} 

\subsection{Module}

\code{classify}

\subsection{Uses}

\begin{itemize}
  \item OAR Model Data Module Specification (\ref{ModuleOMD})
  \item OAR Model Equations Module Specification (\ref{ModuleOME})
\end{itemize}

\subsection{Syntax}

\subsubsection{Exported Constants}

\code{LABELS}:= (A, B, C, D, E, F, G, H, I, J, K, L, M, N, O, P, Q, R, S, T, U, V, W, X, Y, Z)

\subsubsection{Exported Access Programs}

\begin{center}
\begin{tabular}{p{2cm} p{4cm} p{4cm} p{2cm}}
\hline
\textbf{Name} & \textbf{In} & \textbf{Out} & \textbf{Exceptions} \\
\hline
\code{classify} & \code{inputImage} ($\mathbf{U}^{m \times n}$), \code{oarModel} ($\mathbf{R}^{m \times n}$) & \code{resultLabel} (String), \code{confPercent} ($\mathbf{R}_{+}$) & - \\
\hline
\end{tabular}
\end{center}

\subsection{Semantics}

\subsubsection{State Variables}

\begin{itemize}
  \item \code{weight}: the weight portion of the \code{oarModel} input as $\mathbf{M}_{x,y}$ for each label.
  \item \code{bias}: the bias portion of the \code{oarModel} input as $\mathbf{R}$ for each label.
  \item \code{predictionMatrix}: $\mathbf{M}_{x,y}$, where each entry is the output of \code{predict}(inputImage, weight, bias), corresponding to each label
  \item \code{bestPrediction}: \textbf{max}(\code{predictionMatrix}) $\leq 1$
\end{itemize}

\subsubsection{Environment Variables}

None.

\subsubsection{Assumptions}

The input image is valid.

\subsubsection{Access Routine Semantics}

\noindent \code{classify}(inputImage):
\begin{itemize}
\item output: \code{confPercent} := \code{bestPrediction}, 
\code{resultLabel}:= the letter in the list of \code{LABELS} corresponding to the index of \code{bestPrediction} in the \code{predictionMatrix}
\end{itemize}

\subsubsection{Local Functions}

None.

\section{MIS of OAR Model Data Module} \label{ModuleOMD} 

\subsection{Module}

\code{model}

\subsection{Uses}

\begin{itemize}
  \item OAR Model Testing Module Specification (\ref{ModuleOMTs})
\end{itemize}

\subsection{Syntax}

\subsubsection{Exported Constants}

None.

\subsubsection{Exported Access Programs}

\begin{center}
\begin{tabular}{p{2cm} p{4cm} p{4cm} p{2cm}}
\hline
\textbf{Name} & \textbf{In} & \textbf{Out} & \textbf{Exceptions} \\
\hline
\code{model} & - & - & - \\
\hline
\end{tabular}
\end{center}

\subsection{Semantics}

\subsubsection{State Variables}

\begin{itemize}
  \item \code{oarModel}: Data structure designed to store the matrix of weights and biases associated
   with the trained OAR classification model as a tuple of $\mathbf{M}_{x,y}$ and $\mathbf{R}$.
   \item \code{performance}: Data structure designed to store the matrix of performance values associated
   with each label of the trained OAR classification model as a $\mathbf{M}_{x,y}$.
\end{itemize}

\subsubsection{Environment Variables}

None.

\subsubsection{Assumptions}

None.

\subsubsection{Access Routine Semantics}

\noindent \code{model}():
\begin{itemize}
\item transition: This module is a simple tuple ($\mathbf{R}^{m \times n}$ and $\mathbf{R}$) data structure for storing the OAR classification model weights and biases and corresponding performance matrix.
\end{itemize}

\subsubsection{Local Functions}

None.

\section{MIS of OAR Model Equations Module} \label{ModuleOME} 

\subsection{Module}

\code{oarUtils}

\subsection{Uses}

None.

\subsection{Syntax}

\subsubsection{Exported Constants}

None.

\subsubsection{Exported Access Programs}

\begin{center}
\begin{tabular}{p{3cm} p{5cm} p{3cm} p{1cm}}
\hline
\textbf{Name} & \textbf{In} & \textbf{Out} & \textbf{Exceptions} \\
\hline
\code{sigmoid} & \code{sigIn} ($\mathbf{R}$) & \code{sigOut} ($\mathbf{R}$) & - \\
\code{logLossFunc} & \code{trueVal} ($\mathbf{R}$), \code{predVal} ($\mathbf{R}$) & \code{logLoss} ($\mathbf{R}$) & \code{ValueError} \\
\code{predict} & \code{inputImage} ($\mathbf{U}^{m \times n}$), \code{weight} ($\mathbf{R}^{m \times n}$), \code{bias} ($\mathbf{R}$) & \code{predVal} ($\mathbf{R}$) & \code{ValueError} \\
\code{gradientW} & \code{inputImage} ($\mathbf{U}^{m \times n}$), \code{trueVal} ($\mathbf{R}$), \code{weight} ($\mathbf{R}^{m \times n}$), \code{bias} ($\mathbf{R}$), \code{regParam} ($\mathbf{R}$), \code{trainSize} ($\mathbf{Z}_{+}$)& \code{gradW} ($\mathbf{R}$) & \code{ValueError} \\
\code{gradientB} & \code{inputImage} ($\mathbf{U}^{m \times n}$), \code{trueVal} ($\mathbf{R}$), \code{weight} ($\mathbf{R}^{m \times n}$), \code{bias} ($\mathbf{R}$)& \code{gradB} ($\mathbf{R}$) & - \\
\hline
\end{tabular}
\end{center}

\subsection{Semantics}

\subsubsection{State Variables}

None.

\subsubsection{Environment Variables}

None.

\subsubsection{Assumptions}

The input image is valid.

\subsubsection{Access Routine Semantics}

\noindent \code{sigmoid}(sigIn):
\begin{itemize}
\item output: \code{sigOut} := $\frac{1}{1 + e^{-\code{sigIn}}}$
\end{itemize}

\noindent \code{logLossFunc}(trueVal, predVal):
\begin{itemize}
\item output: \code{logLoss} := $\code{trueVal} \cdot \log(\code{predVal}) + (1 - \code{trueVal}) \cdot \log(1 - \code{predVal})$
\item exception: \code{ValueError} if \code{predVal} or $1 -$ \code{predVal} is negative
\end{itemize}

\noindent \code{predict}(inputImage, weight, bias):
\begin{itemize}
\item output: sigIn:= $\code{weight}^{\textbf{T}}\cdot\code{inputImage}+\code{bias}$, 
predVal := \code{sigmoid}(sigIn)
\item exception: \code{ValueError} if \code{inputImage} and \code{weight} are matrices of the same size
\end{itemize}

\noindent \code{gradientW}(inputImage, trueVal, weight, bias, regParam, trainSize):
\begin{itemize}
\item transition: \code{predVal} := \code{predict}(inputImage, weight, bias)
\item output: \code{gradW}:= $\code{inputImage}\cdot(\code{trueVal}-\code{predVal}) - \frac{\code{regParam}}{\code{trainSize}}\cdot\code{weight}^{2}$
\item exception: \code{ValueError} if \code{trainSize} is 0 
\end{itemize}

\noindent \code{gradientB}(inputImage, trueVal, weight, bias):
\begin{itemize}
  \item transition: \code{predVal} := \code{predict}(inputImage, weight, bias)
  \item output: \code{gradB}:= $\code{trueVal}-\code{predVal}$
\end{itemize}

\subsubsection{Local Functions}

None.

\section{MIS of OAR Model Training Module} \label{ModuleOMTr} 

\subsection{Module}

\code{train}

\subsection{Uses}

\begin{itemize}
  \item OAR Model Equations Module Specification (\ref{ModuleOME})
\end{itemize}

\subsection{Syntax}

\subsubsection{Exported Constants}

\begin{itemize}
  \item \code{REG\_PARAM}: The regularization parameter used during model training as $\mathbf{R}_{+}$.
  \item \code{ALPHA}: The learning rate parameter used during model training as $\mathbf{R}_{+}$.
\end{itemize}

\subsubsection{Exported Access Programs}

\begin{center}
\begin{tabular}{p{2cm} p{4cm} p{4cm} p{2cm}}
\hline
\textbf{Name} & \textbf{In} & \textbf{Out} & \textbf{Exceptions} \\
\hline
\code{train} & \code{trainData} ($\mathbf{U}^{m \times n}$),\code{trainLabels} ($\mathbf{Z}_{+}^{m \times n}$), \code{weightBiasMatrix} (tuple of $\mathbf{R}^{m \times n}$ and $\mathbf{R}$), \code{trainSize} ($\mathbf{Z}_{+}$)& \code{weightBiasMatrix} (tuple of $\mathbf{R}^{m \times n}$ and $\mathbf{R}$) & - \\
\hline
\end{tabular}
\end{center}

\subsection{Semantics}

\subsubsection{State Variables}

\begin{itemize}
\item \code{weight}: the weight portion of the \code{weightBiasMatrix} input as $\mathbf{R}^{m \times n}$ for each label.
\item \code{bias}: the bias portion of the \code{weightBiasMatrix} input as $\mathbf{R}$ for each label.
\item \code{gradW}:= \code{gradientW}(trainData, trainLabels, REG\_PARAM, trainSize)
\item \code{gradB}:= \code{gradientB}(trainData, trainLabels, weight, bias)
\end{itemize}

\subsubsection{Environment Variables}

None.

\subsubsection{Assumptions}

None.

\subsubsection{Access Routine Semantics}

\noindent \code{train}(trainData, trainLabels, weightBiasMatrix, trainSize):
\begin{itemize}
\item transition: \code{weight}:= \code{weight} $+$ \code{ALPHA}$\cdot$\code{gradW}, 
\code{bias}:= \code{bias} $+$ \code{ALPHA}$\cdot$\code{gradB}
\item output: \code{weightBiasMatrix}:= (\code{weight},\code{bias})
\end{itemize}

\subsubsection{Local Functions}

None.

\section{MIS of OAR Model Testing Module} \label{ModuleOMTs} 

\subsection{Module}

\code{test}

\subsection{Uses}

\begin{itemize}
  \item OAR Model Data Module Specification (\ref{ModuleOME})
  \item OAR Model Equations Module Specification (\ref{ModuleOME})
  \item OAR Model Training Module Specification (\ref{ModuleOMTr})
  \item Metrics Module Specification (\ref{ModuleCM})
\end{itemize}

\subsection{Syntax}

\subsubsection{Exported Constants}

\begin{itemize}
\item \code{EPOCHS}: The the number of times the model training regression algorithm is ran as $\mathbf{Z}_{+}$.
\item \code{LABELS}:= (A, B, C, D, E, F, G, H, I, J, K, L, M, N, O, P, Q, R, S, T, U, V, W, X, Y, Z) (tuple of Strings)
\item \code{TRAIN\_SIZE}: The size of the training data used during model training as $\mathbf{Z}_{+}$.
\item \code{TEST\_SIZE}: The size of the testing data used during model training as $\mathbf{Z}_{+}$.
\end{itemize}

\subsubsection{Exported Access Programs}

\begin{center}
\begin{tabular}{p{2cm} p{4cm} p{4cm} p{2cm}}
\hline
\textbf{Name} & \textbf{In} & \textbf{Out} & \textbf{Exceptions} \\
\hline
\code{test} & - & \code{oarModel} (tuple of $\mathbf{R}^{m \times n}$ and $\mathbf{R}$), \code{performance} (tuple of $\mathbf{R}^{m \times n}$ and $\mathbf{U}^{m \times n}$) & - \\
\hline
\end{tabular}
\end{center}

\subsection{Semantics}

\subsubsection{State Variables}

\begin{itemize}
  \item \code{dataSize}: The size of the input image matrix used during model training as $\mathbf{Z}_{+}$.
  \item \code{dataSet}: The set of pre-processed images and their associated labels that will be used for training the classification model as a tuple of $\mathbf{I}_{x,y}$ and $\mathbf{Z}_{+}$.
  \item \code{weight}: the weight portion of the \code{weightBiasMatrix} input as $\mathbf{R}^{m \times n}$ for each label.
  \item \code{bias}: the bias portion of the \code{weightBiasMatrix} input as $\mathbf{R}$ for each label.
  \item \code{predictionData}: matrix which tracks the predictions made for each test image as $\mathbf{R}_{+}^{m \times n}$.
\end{itemize}

\subsubsection{Environment Variables}

None.

\subsubsection{Assumptions}

None.

\subsubsection{Access Routine Semantics}

\noindent \code{test}():
\begin{itemize}
\item transition: weightBiasMatrix:= $\mathbf{R}^{m \times n}$ of randomized values from 0 to 1, \code{train}(trainData, trainLabels, weightBiasMatrix, trainSize)
\item output: \code{oarModel}:= weightBiasMatrix, \code{performance}:= \code{confMatrix}(predictionData, trainLabels), \code{loss}(trainLoss).
\end{itemize}

\subsubsection{Local Functions}

\begin{itemize}
\item \code{splitDataSet}(dataSet):
  \begin{itemize}
    \item output: Takes the \code{dataSet} as an input and splits it into distinct parts for training and testing the classification model.
    The following values are output:
    \begin{itemize}
      \item \code{trainData}:= The part of the \code{dataSet} used to train the model as $\mathbf{U}^{m \times n}$.
      \item \code{trainLabels}:= The part of the \code{dataSet} corresponding to the true labels of the \code{trainData} as $\mathbf{Z}_{+}^{m \times n}$.
      \item \code{testData}:= The part of the \code{dataSet} used to test the model as $\mathbf{U}^{m \times n}$.
      \item \code{testLabels}:= The part of the \code{dataSet} corresponding to the true labels of the \code{testData} as $\mathbf{Z}_{+}^{m \times n}$.
    \end{itemize}
  \end{itemize}
\end{itemize}

\section{MIS of Metrics Module} \label{ModuleCM} 

\subsection{Module}

\code{metrics}

\subsection{Uses}

None.

\subsection{Syntax}

\subsubsection{Exported Constants}

None.

\subsubsection{Exported Access Programs}

\begin{center}
\begin{tabular}{p{3cm} p{4cm} p{4cm} p{1cm}}
\hline
\textbf{Name} & \textbf{In} & \textbf{Out} & \textbf{Exceptions} \\
\hline
\code{confMatrix} & \code{predictionData} ($\mathbf{R}_{+}^{m \times n}$), \code{trainLabels} ($\mathbf{Z}_{+}^{m \times n})$ & \code{confusionMatrix} ($\mathbf{U}^{m \times n}$), \code{matrixData} ($\mathbf{Z}_{+}^{m \times n}$)  & - \\
\code{loss} & \code{trainLoss} ($\mathbf{R}^{m \times n}$) & \code{lossGraph} ($\mathbf{U}^{m \times n}$)  & - \\
\hline
\end{tabular}
\end{center}

\subsection{Semantics}

\subsubsection{State Variables}

None.

\subsubsection{Environment Variables}

None.

\subsubsection{Assumptions}

None.

\subsubsection{Access Routine Semantics}

\noindent \code{confMatrix}(predictionData, trainLabels):
\begin{itemize} 
\item output: \code{matrixData}:= ($\sum$\code{predictionData} == \code{True} \& \code{trainLabels} == \code{True}, $\sum$\code{predictionData} == \code{True} \& \code{trainLabels} == \code{False}, $\sum$\code{predictionData} == \code{False} \& \code{trainLabels} == \code{True}, $\sum$\code{predictionData} == \code{False} \& \code{trainLabels} == \code{False}) \\
\code{confMatrix}:= The graphical representation of \code{matrixData}

\end{itemize}

\noindent \code{loss}(trainLoss):
\begin{itemize} 
\item output: \code{lossGraph}:= \code{trainLoss} as a graphical representation over some number of epochs ($\mathbf{Z}_{+})$.
\end{itemize}

\subsubsection{Local Functions}

None.

\section{MIS of Input Processing Module} \label{ModuleIP} 

\subsection{Module}

\code{preprocess}

\subsection{Uses}

None.

\subsection{Syntax}

\subsubsection{Exported Constants}

None.

\subsubsection{Exported Access Programs}

\begin{center}
\begin{tabular}{p{2cm} p{4cm} p{4cm} p{2cm}}
\hline
\textbf{Name} & \textbf{In} & \textbf{Out} & \textbf{Exceptions} \\
\hline
\code{preprocess} & \code{baseImage} ($\mathbf{Z}^{m \times n}$) & \code{inputImage} ($\mathbf{U}^{m \times n}$) & - \\
\hline
\end{tabular}
\end{center}

\subsection{Semantics}

\subsubsection{State Variables}

None.

\subsubsection{Environment Variables}

None.

\subsubsection{Assumptions}

The format and parameters of the base image was already verified to be within the requirements.

\subsubsection{Access Routine Semantics}

\noindent \code{preprocess}(baseImage):
\begin{itemize} 
\item output: Performs transformations on the \code{baseImage} using functions provided by the 
sci-kit learn library, such that the resulting \code{inputImage} 
as ($\mathbf{U}^{m \times n}$), is normalized to be able to be used by the classification model.
\end{itemize}

\subsubsection{Local Functions}

None.

\section{MIS of Graphical User Interface} \label{ModuleGUI} 

\subsection{Module}

\code{gui}

\subsection{Uses}

None.

\subsection{Syntax}

\subsubsection{Exported Constants}

None.

\subsubsection{Exported Access Programs}

\begin{center}
\begin{tabular}{p{2cm} p{4cm} p{4cm} p{2cm}}
\hline
\textbf{Name} & \textbf{In} & \textbf{Out} & \textbf{Exceptions} \\
\hline
\code{gui} & None & None & - \\
\hline
\end{tabular}
\end{center}

\subsection{Semantics}

\subsubsection{State Variables}

None.

\subsubsection{Environment Variables}

\begin{itemize}
  \item Keyboard ($\mathbf{Z}_{+}$ for keycodes describing the key pressed)
  \item Mouse (Boolean for click state and  $\mathbf{Z}_{+}$ for cursor position)
  \item Screen ($\mathbf{Z}_{+}$ for width and height in pixels)
  \item Button (String for a file location) to provide an input image from the file system
\end{itemize}

\subsubsection{Assumptions}

None.

\subsubsection{Access Routine Semantics}

\noindent \code{gui}():
\begin{itemize}
\item transition: Provides methods from the TKinter Library to build and deploy a GUI to Graphics Display Module \ref{ModuleGD}
\end{itemize}

\subsubsection{Local Functions}

None.
  
\newpage

\bibliographystyle {plainnat}
\bibliography {../../../refs/References}


\end{document}