\documentclass{article}

\usepackage{tabularx}
\usepackage{booktabs}

\title{Reflection Report on OAR}

\author{Hunter Ceranic}

\date{April 15, 2024}

%% Comments

\usepackage{color}

\newif\ifcomments\commentstrue %displays comments
%\newif\ifcomments\commentsfalse %so that comments do not display

\ifcomments
\newcommand{\authornote}[3]{\textcolor{#1}{[#3 ---#2]}}
\newcommand{\todo}[1]{\textcolor{red}{[TODO: #1]}}
\else
\newcommand{\authornote}[3]{}
\newcommand{\todo}[1]{}
\fi

\newcommand{\wss}[1]{\authornote{blue}{SS}{#1}} 
\newcommand{\plt}[1]{\authornote{magenta}{TPLT}{#1}} %For explanation of the template
\newcommand{\an}[1]{\authornote{cyan}{Author}{#1}}

%% Common Parts

\newcommand{\progname}{OAR} % PUT YOUR PROGRAM NAME HERE
\newcommand{\authname}{Hunter Ceranic} % AUTHOR NAMES                  

\usepackage{hyperref}
    \hypersetup{colorlinks=true, linkcolor=blue, citecolor=blue, filecolor=blue,
                urlcolor=blue, unicode=false}
    \urlstyle{same}
                                


\begin{document}

\maketitle

\section{Changes in Response to Feedback}

Feedback from the reviewers (Domain Expert, Secondary Reviewer, and the Course Instructor Dr. Smith) was integrated
into the documentation. Namely there was a shift in the projects overall tone to becoming more of a "Educational" tool,
and hopefully that is reflected in the final version of the project.

\subsection{SRS}

Using the SRS document I was able to create a kind of roadmap for where the project was headed, and for me at least, really develop the
themes of the project (those being designing the project to be an educational tool, and using the project as an opportunity to 
teach myself a lot about software development).
Most of the updates over the course of the project related to adding more
references or information to the Theoretical Models (especially since my understanding of them was very weak in the initial revisions),
and updating descriptions or assumptions to be
more clear. The issues created by the reviewers and their corresponding commits that resolve them are found in each issues comments.


\subsection{Design and Design Documentation}

I think the basic component modules that needed to be part of the design were clear from the beginning,
however understanding how those modules needed to be connected, and how to communicate what they did
on a high-level were more difficult. Even in the final version there are clear differences between the design documentation and
how the actual implementation is handled, which I left different intentionally (I think the way its laid out in the documentation
is easier to understand in that context, but it doesn't make as much sense in the actual code). Most of the updates over the course of the project related to adding more
detail to the Module Interface Specification, further modularization of specific parts like th GUI
and updates for clarity. The issues created by the reviewers and their corresponding commits that resolve them are found in each issues comments.

\subsection{VnV Plan and Report}

\section{Design Iteration}


The core of the final design was similar to the initial version, however the way modules were linked to each other
 were updated during the implementation. Module names were changed to match the
file names.

\section{Design Decisions}

The decision to create a GUI for the project was so that I could ensure the proper images were being classified.
The assumptions constrained the problem enough to ensure the project was focused on the goal of making it "educational"
rather then simply a high performance classification model.


\section{Reflection on Project Management}

The specific tools used in this project were CircleCI \cite{CircleCI}, Flake8 \cite{Flake8},
and pytest.

\subsection{How Does Your Project Management Compare to Your Development Plan}

Originally I also planned to use code coverage tools, however due to issues integrating the test suite into CircleCI
these were dropped from the project in interest of time.

\subsection{What Went Well?}

CircleCI was great for testing build portability and Linting. Additionally I learned a lot about CI/CD Markdown, LaTeX, Makefiles, Config files,
and other software development tools during the course of this project, and essentially taught myself how to use a lot of these things
from scratch (which is probably why as the project progress you may also be able to track my competence using these tools).

\subsection{What Went Wrong?}

Integrating CircleCI and pytest was a pain. There was a lot of issues when it came to understanding how to link all the files properly and
have the tests be recognized by CircleCI. It took way too long to figure that out and as a result a lot of the testing suffered, since I simply
did not have enough time to cover it all. As a result, I dropped doing the Performance Non-Functional Requirement Tests, as the actual
performance of the model was least important to my learning overall, and also to the purpose of the project.

\subsection{What Would you Do Differently Next Time?}

Next time I would probably use GitHub Actions. From what other have said it seems a bit simpler to integrate pytest with, and that would save me a 
lot of time., and give me the ability to improve the quality of the software and documentation I wrote and the testing performed.

\end{document}