\documentclass{article}

\usepackage{tabularx}
\usepackage{booktabs}

\title{Problem Statement and Goals\\Optical Alphabet Recognition}

\author{\textbf{Hunter Ceranic}}

\date{January 19, 2024}

\input{../Comments}
%% Common Parts

\newcommand{\progname}{OAR} % PUT YOUR PROGRAM NAME HERE
\newcommand{\authname}{Hunter Ceranic} % AUTHOR NAMES                  

\usepackage{hyperref}
    \hypersetup{colorlinks=true, linkcolor=blue, citecolor=blue, filecolor=blue,
                urlcolor=blue, unicode=false}
    \urlstyle{same}
                                


\begin{document}

\maketitle

\begin{table}[hp]
\caption{Revision History} \label{TblRevisionHistory}
\begin{tabularx}{\textwidth}{llX}
\toprule
\textbf{Date} & \textbf{Developer(s)} & \textbf{Change}\\
\midrule
January 19, 2024 & Hunter Ceranic & Creation of Document\\
\bottomrule
\end{tabularx}
\end{table}

\section{Problem Statement}
Optical character recognition is a well known computer vision problem that uses various forms of image classification 
convert image data into understandable text. Optical character recognition is widely used in many different fields such as 
finance, healthcare and supply chain management and as such many tools have been developed to streamline ease of 
access and use of the technology. This project aims to create clear, in-depth documentation outlining
the creation of an optical character recognition program, specifically for capitalized alphabet characters, using 
an image classification model to translate images into the respective alphabet characters they represent.

\wss{You should check your problem statement with the
\href{https://github.com/smiths/capTemplate/blob/main/docs/Checklists/ProbState-Checklist.pdf}
{problem statement checklist}.}
\wss{You can change the section headings, as long as you include the required information.}

\subsection{Problem}

\subsection{Inputs and Outputs}
Inputs to the program will be images of characters to be identified.
Outputs of the program will be a probability that the image depicts one of the letters in the 
set of {A, B, C, D, E, F, G, H, I, J, K, L, M, N, O, P, Q, R, S, T, U, V, W, X, Y, Z}.
\wss{Characterize the problem in terms of ``high level'' inputs and outputs.  
Use abstraction so that you can avoid details.}

\subsection{Stakeholders}

\subsection{Environment}

\wss{Hardware and Software}

\section{Goals}

\section{Stretch Goals}

\end{document}