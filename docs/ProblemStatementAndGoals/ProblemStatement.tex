\documentclass{article}

\usepackage{tabularx}
\usepackage{booktabs}

\title{Problem Statement and Goals\\Optical Alphabet Recognition}

\author{\textbf{Hunter Ceranic}}

\date{February 19, 2024}

%% Comments

\usepackage{color}

\newif\ifcomments\commentstrue %displays comments
%\newif\ifcomments\commentsfalse %so that comments do not display

\ifcomments
\newcommand{\authornote}[3]{\textcolor{#1}{[#3 ---#2]}}
\newcommand{\todo}[1]{\textcolor{red}{[TODO: #1]}}
\else
\newcommand{\authornote}[3]{}
\newcommand{\todo}[1]{}
\fi

\newcommand{\wss}[1]{\authornote{blue}{SS}{#1}} 
\newcommand{\plt}[1]{\authornote{magenta}{TPLT}{#1}} %For explanation of the template
\newcommand{\an}[1]{\authornote{cyan}{Author}{#1}}

%% Common Parts

\newcommand{\progname}{OAR} % PUT YOUR PROGRAM NAME HERE
\newcommand{\authname}{Hunter Ceranic} % AUTHOR NAMES                  

\usepackage{hyperref}
    \hypersetup{colorlinks=true, linkcolor=blue, citecolor=blue, filecolor=blue,
                urlcolor=blue, unicode=false}
    \urlstyle{same}
                                


\begin{document}

\maketitle

\begin{table}[hp]
\caption{Revision History} \label{TblRevisionHistory}
\begin{tabularx}{\textwidth}{llX}
\toprule
\textbf{Date} & \textbf{Developer(s)} & \textbf{Change}\\
\midrule
January 19, 2024 & Hunter Ceranic & Creation of Document\\
February 19, 2024 & Hunter Ceranic & Revision after feedback from Dr. Smith\\
\bottomrule
\end{tabularx}
\end{table}

\section{Problem Statement}
Optical character recognition is a well known computer vision problem that uses image classification methods and algorithms to
convert image data into understandable text. Optical character recognition is widely used in many different fields such as 
finance, healthcare and supply chain management to help identify important communications and written data in images. 
As such many tools and programming libraries have been developed to streamline ease of access and use of the technology. 
This project aims to create clear, in-depth documentation outlining the creation of an optical character recognition program,
specifically for capitalized English alphabet characters, using an image classification model to translate images into the 
respective alphabet characters they represent.

\subsection{Problem}
The problem that is being addressed by this project is the creation of an image classification algorithm to convert 
pixel data from an image of an upper-case letter to the corresponding text chararacter.
\subsection{Inputs and Outputs}
Inputs to the program will be pixel data from images of characters to be identified.

Outputs of the program will be a probability that the image depicts one of the English alphabet letters in the 
set of \{A, B, C, D, E, F, G, H, I, J, K, L, M, N, O, P, Q, R, S, T, U, V, W, X, Y, Z\}.

\subsection{Environment}
This will be a cross-platform (ie. Windows, Linux or Mac) program able to operate on any modern laptop or desktop.

\section{Goals}
The end goal of this project is to accurately identify capital letters consistently, with a reasonable 
amount of confidence.
\section{Stretch Goals}
Furthermore, beyond the intended scope the project, if possible a stretch goal would be improving
the program so that it would be useable for all other printing ASCII characters.

\end{document}